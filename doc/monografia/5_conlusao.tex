\chapter{Conclusão} \label{chapter:conclusion}

\section{Contribuições}

A principal contribuição deste trabalho é a biblioteca \textit{Hidra}, criada com o apoio da arquitetura de referência \textit{Cambuci}\cite{dissertacaoOsshiro2014}, e que possibilitará a criação de repositórios de ativos de software, podendo futuramente ser extendida. Assim como resaltado em (OSSHIRO, 2014) essa arquitetura de referência contribui efetivamente com a área de reúso de software, pois pode ser especializada para qualquer sub-domínio do domínio de repositórios de ativos de software, como repositórios de LPN, repositórios de requisitos, repositórios de modelos de análise, repositórios de teste de software, entre outros. Adicionalmente também contribuirá com a abordagem GLPN (Landre, 2012), visto que possibilitará a construção de um repositório de LPN.

\section{Trabalhos Futuros}


A biblioteca \textit{Hidra} esta em sua versão inicial, e não aborda todos os aspectos de granularidade de um ativo de software reusável, possibilitando ainda o desenvolvimento de trabalhos futuros dos quais listamos alguns:

\begin{itemize}

\item Extensão da biblioteca permitindo a definição e implementação dos Perfils de um ativo de Software de acordo com o modelo proposto pela \textit{(OMG)}.

\item Extensão da biblioteca de modo a comtemplar especificidades relacionadas a LPNs.

\item Extensão da biblioteca de modo a fornecer mecanismos de buscas de ativos utilizando algoritmos de busca.

\item Extensão da camada de serviços da biblioteca de modo a forncer informações sobre suas caracteristicas normativas de uso, por meio de descrições padronizadas seguindo o padrão DNS para descoberta de serviços.

\item Extensão da camada de serviços da biblioteca de modo a viabilizar as descrições semânticas de um repositório, permitindo assim a sua classificação.

\item Extensão da biblioteca hidra de modo a viabilizar o desenvolvimento de repositórios que tenham a disposição informaões e documentos relacionados às suas caracteristicas de qualidade.

\item Extensão da biblioteca hidra de modo a prover mecanismos para a captura, monitoramento, registro e sinalização do não cumprimento de requisitos de qualidade estabelecidos entre serviços provedores e serviços clientes.

\end{itemize}