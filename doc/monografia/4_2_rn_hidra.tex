\begin{longtable}{ | l | p{4cm} | p{2cm} | p{6cm} |}
\caption{Requisitos Não-Funcionais Hidra}\\
\hline
\textbf{ID} & \textbf{Requisito} & \textbf{RA Cambuci} & \textbf{Solução}  \\
\hline
\endfirsthead
\multicolumn{4}{c}%
{\tablename\ \thetable\ -- \textit{Requisitos Não-funcionais Hidra}} \\
\hline
\textbf{ID} & \textbf{Requisito} & \textbf{RA Cambuci} & \textbf{Solução}  \\
\hline
\endhead
\hline \multicolumn{4}{r}{\textit{Continua na página seguinte}} \\
\endfoot
\hline
\endlastfoot
	RN-01
	& A biblioteca \textit{Hidra} deve fornecer mecanismos para que repositórios de ativos de software aceitem múltiplas fontes de origem de ativos, por meio de serviços web que possam ser publicados, localizados e utilizados de maneira uniforme.
	& RA-AS[9],

	RAS[3],

	RAS[4], 

	RAS[10]
	& Requisito é atendido por meio da camada de serviço provida pela biblioteca, que segue o padrão REST, que permitirá ao repositório desenvolvido, tendo como base a \textit{Hidra}, fácil acesso e integração a múltiplas ferramentas.
	\\ \hline

	RN-02
	& A biblioteca \textit{Hidra} deve fornecer mecanismos de versionamento aos ativos de software.
	& RA-AS[10]
	& Requisito é atendido por meio da camada de persistência provida pela biblioteca, que utiliza a API jGit para manipulação das operações de Gerenciamento de Configuração sobre um repositório Git, e permitirá que o repositório armazene múltiplas versões de um mesmo ativo.
	\\ \hline

	RN-03
	& A biblioteca \textit{Hidra} deve oferecer mecanismos  para  gerenciamento da configuração de ativos de software.
	& RA-AS[11] 
	& Requisito é atendido por meio da camada de persistência provida pela biblioteca, que utiliza a API jGit para manipulação das operações de Gerenciamento de Configuração sobre um repositório Git, e permitirá que o repositório gerencie configurações de ativos de software.
	\\ \hline

	RN-04
	& A biblioteca \textit{Hidra} permitir que repositórios garantam que seus ativos de software não sofram alterações não autorizadas.
	& RA-AS[16],

	RAS[4]
	& Requisito é atendido por meio do controle de usuários da camada de persistência provida pela biblioteca, que utiliza a API jGit para manipulação das operações de Gerenciamento de Configuração sobre um repositório Git. Na versão inicial, a biblioteca utiliza um usuário padrão informado no seu arquivo de propriedades (hidra.properties).
	\\ \hline

	RN-05
	& A biblioteca \textit{Hidra} deve ser extensível de modo a viabilizar o desenvolvimento de repositório de ativos de software escalável, capaz de evoluir de maneira incremental, por meio da composição de novas funcionalidades disponíveis na forma de serviços.
	& RAS[9] 
	& Requisito é atendido por meio dos padrões adotados para a implementação da biblioteca: i) Padrão RAS para representação e manipulação de Ativos de Software Reusáveis; ii) Divisão dos recursos providos em camadas (jGit para persistência, Hidra para regras de negócio, HidraService para fornecimento de serviços; iii) Padrões de Projeto (tanto padrões GRASP quanto padrões GoF) adotados na implementação, como por exemplo, Singleton, Facade, Strategy, Especialista na Informação).
 \\ \hline 

\end{longtable}
