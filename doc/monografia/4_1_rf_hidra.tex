\begin{longtable}{ | l | p{6cm} | p{2cm} | p{4cm} |}
\caption{Requisitos Funcionais Hidra}\\
\hline
\textbf{ID} & \textbf{Requisito} & \textbf{RA Cambuci} & \textbf{Solução}  \\
\hline
\endfirsthead
\multicolumn{4}{c}%
{\tablename\ \thetable\ -- \textit{Requisitos Funcionais Hidra}} \\
\hline
\textbf{ID} & \textbf{Requisito} & \textbf{RA Cambuci} & \textbf{Solução}  \\
\hline
\endhead
\hline \multicolumn{4}{r}{\textit{Continua na página seguinte}} \\
\endfoot
\hline
\endlastfoot
	RF-01
	& A biblioteca \textit{Hidra} deve permitir a \textbf{inclusão de ativos de software}, levando em consideração a composição de um ativo por diferentes artefatos.
	& RA-AS[1]
	& Os ativos reusáveis de software são armazenados no repositório em forma de diretórios, por meio, da segunda forma de armazenamento RAS \cite{omg2005}. \\ \hline

	% 4_1_rf_hidra.tex
	RF-02
	& A biblioteca \textit{Hidra} deve fornecer mecanismos a fim de listar artefatos que compõem um ativo de software armazenado no repositório.
	& RA-AS[1]
	& Requisito implementado por meio dos métodos \textit{Asset.getSolution()} e \textit{Asset.setSolution()}. \\ \hline

	RF-03
	& A biblioteca \textit{Hidra} deve possuir uma estrutura padronizada de representação, comunicação e armazenamento de ativos de software. 
	& RA-AS[2] 
	& Foi adotado o padrão RAS atualmente em sua versão 2.2. \\ \hline

	RF-04
	& A biblioteca \textit{Hidra} deve garantir que todo novo ativo de software seja validado e certificado de acordo com o padrão adotado.
	& RA-AS[2]
	& As regras especificadas no padrão RAS, expressas em forma de um XSD \textit{NomeArquivoXSD.xsd}, são consultadas ao validar e certificar um Ativo antes de qualquer atualização ou inserção (método \textit{Asset.validate()}).\\ \hline

	RF-05
	& A biblioteca \textit{Hidra} deve garantir que ativos de software, que não forem mais utilizados, possam ser removidos do repositório.
	& RA-AS[3]
	& Requisito implementado por meio do método \textit{Repository .removeAsset(Asset asset)}. \\ \hline

	RF-06
	& A biblioteca \textit{Hidra} deve possibilitar a adição de informações para classificação de um ativo e também o contexto de sua utilização.
	& RA-AS[4] 
	& Requisito implementado por meio dos métodos \textit{Asset.getClassification()} e \textit{Asset.setClassification()}. \\ \hline

	RF-07
	& A biblioteca \textit{Hidra} deve possibilitar a adição de informações sobre regras para instalação, personalização, e utilização do ativo.
	& extensão do requisito RA-AS[4] baseando-se no padrão RAS.
	& Requisito implementado por meio dos métodos \textit{Asset.getUsage()} e \textit{Asset.setUsage()}. \\ \hline

	RF-08
	& A biblioteca \textit{Hidra} deve possibilitar o registro de dependência entre ativos.
	& RA-AS[5] 
	& Requistio implementado por meio dos métodos \textit{Asset.getRelatedAssets()} e \textit{Asset.setRelatedAsset()}. \\ \hline

	RF-09
	& A biblioteca \textit{Hidra} deve oferecer informações relevantes a todos os interessados, sobre mudanças que aconteçam no ativo de software: data de alteração, autor da alteração, o que foi alterado e descrição sobre a alteração.
	& RA-AS[6] 
	& Requisito implementado por meio do método \textit{Asset.getLog()} \\ \hline

	RF-10
	& A biblioteca \textit{Hidra} deve fornecer mecanismos a fim de listar ativos armazenados no repositório.
	& RA-AS[7]
	& Requisito implementado por meio do método \textit{Repository.listAssets()} \\ \hline

	RF-11
	& A biblioteca \textit{Hidra} deve fornecer mecanismos a fim de recuperar um ativo armazenado no repositório (download).
	& RA-AS[7] 
	& Requisito implementado por meio do método \textit{Repository .retrieveAsset()} \\ \hline

	RF-12
	& A biblioteca \textit{Hidra} deve fornecer mecanismos a fim de buscar ativos armazenados no repositório.
	& RA-AS[7] 
	& Requisito não implementado na versão atual da biblioteca \textit{Hidra} (Trabalho Futuro). \\ \hline

	RF-13
	& A biblioteca \textit{Hidra} deve fornecer mecanismos que garantem a atomicidade, consistência e isolamento de transações de controle de ativos de software
	& RA-AS[17]
	& A biblioteca provê recursos para que as transações sejam controladas considerando os aspectos citados: os recursos da API jGit e a camada de serviços. Mas essa implementação deverá ser realizada diretamente no repositório. \\ \hline

	RF-14
	& A biblioteca \textit{Hidra} deve fornecer mecanismos que permitem a persistência de diferentes tipos de ativos.
	& RAS[1]
	& O padrão RAS, adotado na biblioteca \textit{Hidra} para a implementação dos requisitos relacionados ao Ativo de Software, permite a persistência de diferentes tipos de ativos, desde que a estrutura de cada ativo seja descrita em sua solução (métodos \textit{Asset.getSolution()} e \textit{Asset.setSolution()}). \\ \hline

	RF-15
	& A biblioteca \textit{Hidra} deve fornecer mecanismos que permitem a persistência ativos implementados em diferentes linguagens de programação.
	& RAS[2] 
	& O padrão RAS, adotado na biblioteca \textit{Hidra} para a implementação dos requisitos relacionados ao Ativo de Software, permite a persistência de ativos implementados em diferentes linguagens de programação, desde que as regras para instalação, personalização, e utilização de cada ativo seja descrita na especificação de seu uso (métodos \textit{Asset.getUsage()} e \textit{Asset.setUsage()}). \\ \hline

	RF-16
	& A biblioteca \textit{Hidra} deve fornecer uma camada de serviços (Webservice) com informações sobre suas características e direções normativas de uso, por meio de descrições padronizadas seguindo o padrão DNS para descoberta de serviços.
	& RAS[5]
	& Requisito não implementado na versão atual da biblioteca \textit{Hidra} (Trabalho Futuro). \\ \hline

	RF-17
	& A biblioteca \textit{Hidra} deve viabilizar o desenvolvimento de um repositório de ativos de software com uma camada webservice com descrições semânticas, permitindo assim sua classificação nos respositórios de serviços.
	& RAS[6]
	& Requisito não implementado na versão atual da biblioteca \textit{Hidra} (Trabalho Futuro). \\ \hline

	RF-18
	& A biblioteca \textit{Hidra} deve viabilizar o desenvolvimento  de repositório de ativos de software que tenham à disposição  informações e documentos relacionados às suas características  de qualidade.
	& RAS[7]
	& Requisito não implementado na versão atual da biblioteca \textit{Hidra} (Trabalho Futuro). \\ \hline

	RF-19
	& A arquitetura de referência deve prover mecanismos para a  captura, monitoramento, registro e sinalização do não  cumprimento de requisitos de qualidade estabelecidos entre  serviços provedores e serviços clientes.
	& RAS[8]
	& Requisito não implementado na versão atual da biblioteca \textit{Hidra} (Trabalho Futuro). \\ \hline


\end{longtable}
