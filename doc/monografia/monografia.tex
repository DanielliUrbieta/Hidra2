\documentclass[11pt,a4paper]{report}
\usepackage[portuguese,brazil]{babel}
\usepackage[utf8]{inputenc}
\usepackage[T1]{fontenc}
\usepackage{graphicx}
\usepackage{indentfirst}
\usepackage[none]{hyphenat}
\usepackage{multirow}
\usepackage{longtable}
\usepackage{wrapfig}
\usepackage{float}



%\usepackage{glossary}
%\makeglossary

%\usepackage{verbatim}

% fonts
%\usepackage{palatino}
%\usepackage{courier}

% math
\usepackage{amsmath, amsthm, amssymb}
\usepackage{latexsym}
\newtheorem{teor}{Teorema}
\newtheorem{definicao}[teor]{Definição}
\newtheorem{equacao}[teor]{Equação}

% algorithms
\usepackage[ruled,chapter]{algorithm}
\usepackage[]{algpseudocode}
\algrenewcommand{\algorithmiccomment}[1]{\hskip0em$\rightarrow$ #1}
\renewcommand{\listalgorithmname}{Lista de Algoritmos}

% hiperlinks
\usepackage[pdfpagemode=UseOutlines,colorlinks=true,
a4paper,breaklinks=true,hyperindex,linkcolor=black,
anchorcolor=black,citecolor=black,filecolor=magenta,
menucolor=black,urlcolor=blue,bookmarks=true,
bookmarksopen=true,pdfpagelayout=SinglePage,
pdfpagetransition=Dissolve]{hyperref}

% format
\setlength{\parskip}{0.2cm}
\setlength{\textwidth}{17.0cm}
\setlength{\textheight}{25.3cm}
\setlength{\hoffset}{-2.0cm}
\setlength{\voffset}{-2.0cm}

% blank page
\newcommand{\blankpage}{
\newpage
\thispagestyle{empty}
\mbox{}
\newpage
}

\sloppy
\begin{document}

\pagestyle{empty}
\input capa.tex
\newpage

%\blankpage

\renewcommand{\abstractname}{Resumo}
\begin{abstract}
\thispagestyle{plain}
\pagenumbering{roman}
\setcounter{page}{2}


A reutilização de softwares esta amplamente ligada a engenharia de software, ao momento em que soluções são obtidas estas são novamente utilizadas para a resolução de novos problemas similares.

Uma peça essencial para o gerenciamento e controle dos componentes que comtemplam tais soluções são os repositórios, que auxiliam a garantia da qualidade no processo de desenvolvimento de softwares, visto que uma solução armazenada em um repositório padronizado condiz com o mínimo de qualidade necessária para sua utilização, ou seja, seus artefatos passaram por um processo de validação, e testes. 

É conhecida a existência de diferentes tipos de repositórios, já que não só de codigo fonte são formadas as soluções para um produto, mas também de documentação, padrões de projetos, modelos e inclusives produtos similares, e o desenvolvimento de repositórios seguindo uma formalização, facilita a sistemática utilizada para a obtenção e usuabilidade de artefatos.

O desenvolvimento de um repositório padronizado necessita de uma análise que resulta em orientações para a sua criação e de mecanismos que descrevam o seu funcionamento. Através de uma arquitetura anteriormente estabelecida junto a um escopo e definição de uma estrutura representativa a uma solução, inicia-se o desenvolvimento das funcionalidades de um repositório, como por exemplo, a validação e aceitação do tipo de solução que será armazenada no repositório. Estamos nos referindo a ativos de software reusáveis e o objeto de estudo deste trabalho é composto pela criação de funcionalidades de um repositório, no caso, uma biblioteca que auxilie a criação de repositórios de ativos de software que seguem o padrão determinado pelo modelo \textit{RAS (OMG 2005)}.

Uma vez criada, a biblioteca auxiliará no desenvolvimento de repositórios de ativos de modo que permitirá a descoberta de suas funcinalidades por meio de serviços web facilitando a integração entre diferentes repositórios de ativos de software.

\vspace{1cm}
\hspace{-\parindent}Palavras-chave: Java, REST, Webservices, Repositório, Ativos Reusáveis de Software.
\end{abstract}

%\blankpage

\renewcommand{\abstractname}{Abstract}
\begin{abstract}
\thispagestyle{plain}
\pagenumbering{roman}
\setcounter{page}{3}


The software reuse is largely linked to software engineering, the time at which solutions are obtained, these are again used to solve new problems similar. An essential piece for the management and control of components that comtemplam such solutions are the repositories that help ensure quality in the software development process, as a solution stored in a standardized repository, consistent with the minimum quality required for their use, i.e., their artifacts passed through a validation process, and testing. It is known the existence of different types of repositories, as not only the source code solutions are formed for a product, but also documentation, design patterns, models and inclusives similar products, and the development of repositories following a formalization facilitates systematically used to obtain and Usuability artifacts.

The development of a standardized repository requires an analysis that results in guidelines for their creation and mechanisms that describe its operation. Using a previously established architecture along with a scope and definition of a representative frame to a solution initiated the development of features of a repository, such as the validation and acceptance of the type of solution which is stored in the repository. We are referencing the reusable software assets, and the object of study of this work consists of the creation of features of a repository, if a library that helps to create repositories of software assets that follow the pattern determined by the model \textit{RAS (OMG 2005)}. Once created, the library will assist in the development of repositories of assets that will enable the discovery of their funcinalidades through web services facilitating integration between different repositories of software assets.

\vspace{1cm}
\hspace{-\parindent}Keywords: Java, REST, Webservices, Repository, Software Reusable Assets.
\end{abstract}

%\blankpage

\pagestyle{plain}
\pagenumbering{roman}
\setcounter{page}{4}

\tableofcontents
\listoffigures
\listoftables
\newpage

\pagestyle{plain}
\pagenumbering{arabic}
\setcounter{page}{1}

\chapter{Introdução} \label{chapter:introducao}

Texto.

\section{Seção} \label{section:sec1}

Caso seja necessário dividir em seções.
\chapter{Embasamento Teórico e Trabalhos Relacionados} \label{chapter:embasamento}

Texto.

\section{Seção} \label{section:sec1}

Caso seja necessario dividir o texto do capitulo em secoes.
\chapter{Tecnologias e Ferramentas} \label{chapter:ferramentas}

\section{Descrição das tecnologias e ferramentas utilizadas}
Esta seção apresenta as ferramentas e tecnologias utilizadas no desenvolvimento do projeto \textit{Hidra}.



\begin{itemize}
\item \textbf{NetBeans IDE (Integrated Development Environment)}. Ambiente de desenvolvimento utilizado. Ferramenta para que permite escrever, compilar, depurar e instalar programas. O IDE é completamente escrito em Java, mas também da suporte a qualquer linguagem de programação. possui um grande número de módulos para extender suas funcionalidades. O NetBeans IDE é livre, sem restrições à sua forma de utilização.

\item \textbf{JDK 8 (Java Development Kit)}. Inclui ferramentas úteis para desenvolver e testar os programas escritos na linguagem de programação Java e em execução na plataforma Java.

\item \textbf{JAXB (Java Architecture for XML Binding)}. \cite{dissertacaoHenriqueFaria2005} Auxilia na manipulação de documentos XML, através da linguagem de programação Java. Ao invés de percorrer o documento, separá-lo logicamente em partes discretas, e repassar o conteúdo para a aplicação Java (ORT, 2003), assim como o SAX (SAX, 2005) ou o DOM (W3C, 2005), o JAXB funciona em duas etapas.
    \begin{itemize}
    \item Um esquema XML é compilado em códigos fonte de classes.
    
    \item Após o mapeamento (\textit{binding}) de um esquema XML em um conjunto de classes, então, pode-se, através de uma aplicação que utilize as bibliotecas do JAXB e as classes geradas pelo compilador:
        \begin{itemize}

        \item criar uma árvore de objetos em memória a partir de um documento XML (\textit{unmarshalling});
        
        \item criar um documento XML a partir de objetos Java (\textit{marshalling});
        
        \end{itemize} 
    \end{itemize}
    

\item \textbf{JGit}. Biblioteca Java do sistema de controle de versão Git, na qual contém.
    \begin{itemize}
    \item rotinas de acesso ao repositório
    
    \item protocolos de rede
    
    \item principais algoritmos de controle de versão
    \end{itemize}

\item \textbf{Apache Maven}. Ferramenta de automação para compilação utilizada em projetos Java,  Similar à ferramenta Ant, hospedada pela Apache Software Foundation. Utiliza-se de um arquivo XML (\textit{POM}) para descrever o projeto de software sendo construído, suas dependências sobre módulos e componentes externos, a ordem de compilação, diretórios e plug-ins necessários. O maven Importa bibliotecas Java e seus plug-ins dinamicamente de um ou mais repositórios online, como o \textit{Maven 2 Central Repository}.


\item \textbf{java.util.Properties}. Classe Java que permite armazenar informações indexadas por uma palavra chave. A partir dela é possível fazer uma rápida pesquisa para recuperar informações necessárias em aplicativos Java SE e Java EE. As informações armazenadas por esta classe podem permanecer em memória ou ser persistida em um arquivo texto.

\item \textbf{Padrões de Projeto Grasp e Grop}. \cite{larman2004} Utilização de padrões Facade e Especialista na Informação.

\textbf{Facade}: Classe Facade para simplificar o acesso a subconjuntos de métodos e classes, diminuição do acoplamento, e melhor manutenibilidade.

\textbf{Especialista na Informação}: Padrão utilizado para atribuição de responsabilidades. As responsabilidades serão atribuidas a quem realmente detêm a informação necessária para preencher os requisitos daquela responsabilidade.

\end{itemize}
\chapter{Hidra} \label{chapter:hidra}

Texto.

\section{Requisitos Funcionais} \label{section:sec1}

A abordagem para elaboção dos requisitos funcionais da biblioteca \textit{ Hidra} utiliza-se de derivações do requisitos funcionais contidos na arquitetura de referência Cambuci-LPN (Seção: Trabalhos Relacionados.) em novos requisitos voltados diretamente para a definição da biblioteca \textit{ Hidra}.

A tabela a seguir representa os requisitos funcionais originais da arquitetura de referência Cambuci-LPN, juntamente com seus respectivos identificadores. Os novos requisitos funcionais pertencentes a biblioteca \textit{ Hidra} estão contidos na coluna referênciada por "Requisito Derivado"


\newpage
\begin{longtable}{ | l | p{6cm} | p{6cm} |}
\caption{Tabela de Requisitos Hidra}\\
\hline
\textbf{ID} & \textbf{Requisito Original} & \textbf{Requisito Derivado}  \\
\hline
\endfirsthead
\multicolumn{3}{c}%
{\tablename\ \thetable\ -- \textit{Tabela de Requisitos Hidra}} \\
\hline
\textbf{ID} & \textbf{Requisito Original} & \textbf{Requisito Derivado}  \\
\hline
\endhead
\hline \multicolumn{3}{r}{\textit{Continua na página seguinte}} \\
\endfoot
\hline
\endlastfoot
  RA-AS[1]
  & A arquitetura de referência deve possibilitar que repositórios de ativos de software incluam um novo ativo, que pode ser composto por vários artefatos.
  & A biblioteca de controle deve permitir a inclusão de ativos de software levando em consideração a composição de um ativo por diferentes artefatos.

A biblioteca de controle de controle deve fornecer mecanismos a fim de listar artefatos que compõe um ativo de processo. \\ \hline
    
    RA-AS[2] 
    & A arquitetura de referência deve possibilitar que repositórios de ativos de software forneçam mecanismo para aceitação e certificação de ativos.
    & A biblioteca de controle deve ser capaz de possuir uma estrutura de representação de ativos de software, com finalidade de definir um padrão para o controle de ativos.
A biblioteca de controle de ativos deve garantir que todo novo ativo de software seja validado e certificado de acordo com o padrão RAS. \\ \hline

     RA-AS[3]
     & A arquitetura de referência deve possibilitar que repositórios de ativos de software desativem ativos que não serão mais utilizados.
     & A biblioteca de controle deve garantir que ativos de software, que não forem mais utilizados, possam ser removidos do repositório. \\ \hline
     
    RA-AS[4] 
    & A arquitetura de referência deve possibilitar que repositórios de ativos de software permitam a classificação de um ativo e também informar o contexto de sua utilização.
    & A biblioteca de controle de ativos deve possibilitar a adição de informações para classificação de um ativo e também o contexto de sua utilização. \\ \hline

     RA-AS[5] 
     & A arquitetura de referência deve possibilitar que repositórios de ativos de software registrem a dependência entre ativos.
     & A biblioteca de controle deve possibilitar a descrição dos ativos relacionados por meio de atributos pertencentes a estrutura representante do ativo.
 \\ \hline

    RA-AS[6] 
    & A arquitetura de referência deve possibilitar que repositórios de ativos de software notifiquem os interessados sobre mudanças que aconteçam no ativo. 
    & A biblioteca deve oferecer informações relevantes a todos os interessados, sobre mudanças que aconteçam no ativo de software, como por exemplo, data de alteração e autor da alteração.
 \\ \hline
 
    RA-AS[7] 
    & A arquitetura de referência deve possibilitar que  repositórios de ativos de software permitam realizar  buscas e recuperação dos ativos 
    & A busca e recuperação de ativos não será abordada dentro do escopo inicial do desenvolvimento da biblioteca Hidra, podendo ser implementada futuramente. ) \\ \hline
    
    RA-AS[8] 
    & A arquitetura de referência deve possibilitar que  repositórios de ativos de software permitam a  navegação entre ativos 
    & A abordagem inicial do desenvolvimento da biblioteca hidra considera a navegação entre ativos de software pertencente a um escopo futuro, não estando incluso de primeira estancia) \\ \hline
    
    RA-AS[9] 
    & A arquitetura de referência deve possibilitar que  repositórios de ativos de software aceite múltiplas  fontes de origem de ativos, com o objetivo de facilitar  a integração entre equipes e entre repositórios  diferentes.  
    & A biblioteca de controle  deve fornecer mecanismos de controle e validação de ativos de software oriundos de fontes externas, por meio de serviços padronizados de integração.
 \\ \hline
 
    RA-AS[10]
    & A arquitetura de referência deve possibilitar que  repositórios de ativos de software criem e armazenem  múltiplas versões de um mesmo ativo.
    & A biblioteca de controle deve fornecer mecanismos de versionamento aos ativos de software. \\ \hline
    
    RA-AS[11] 
    & A arquitetura de referência deve possibilitar que  repositórios de ativos de software gerencie a  configuração, como por exemplo, a definição dos itens  do ativo que são configuráveis, o controle de  mudanças dos itens do ativo que são configuráveis.
    & A biblioteca de controle deve oferecer mecanismos  para  gerenciamento da configuração de ativos de software.  \\ \hline
    
    RA-AS[12] 
    &A arquitetura de referência deve possibilitar que  repositórios de ativos de software permita o registro de  impressões dos usuários a respeito da versão do ativo  que eles utilizaram. 
    & O escopo inicial do desenvolvimento da biblioteca hidra tem como foco os requisitos fundamentais de repositorio de ativos de software, transportando o requisito RA-AS[12] para uma abordagem futura em uma nova análise de escopo \\ \hline
    RA-AS[13] 
    & A arquitetura de referência deve possibilitar que  repositórios de ativos de software registrem métricas  coletadas sobre a utilização do ativo.
    & Não condiz com contexto da biblioteca Hidra, uma vez que que o foco da implementação que não visa
a elaboração de métricas. \\ \hline
    RA-AS[14] 
    & A arquitetura de referência deve possibilitar que  repositórios de ativos de software ofereçam  informações relativas ao reúso, iniciativas de reúso,  ativos mais usados, etc.
    & O escopo inicial do desenvolvimento da biblioteca hidra tem como foco os requisitos fundamentais de repositorio de ativos de software, transportando o requisito RA-ASS[14] para uma abordagem futura em uma nova análise de escopo. \\ \hline
    
    RA-AS[15] 
    & A arquitetura de referência deve possibilitar que  repositórios de ativos de software permitam o acesso de  acordo com o papel que o usuário assume.
    & A implementação inicial da biblioteca hidra não abrange o escopo de controle de permissão de usuários. \\ \hline
    
    RA-AS[16] & A arquitetura de referência deve possibilitar que  repositórios de ativos de software garantam a  integridade dos ativos, ou seja, que eles não sofram  alterações não autorizadas.
    & A biblioteca de controle deve garantir que o repositório remoto e principal não sofra alterações não autorizadas.
 \\ \hline
    RA-AS[17] 
    & A arquitetura de referência deve possibilitar que  repositórios de ativos de software realizem o  gerenciamento de transação, garantindo a atomicidade,  consistência, isolamento e durabilidade.
    & A biblioteca de controle deve fornecer mecanismos que garantem a atomicidade, consistência e isolamento de transações de controle de ativos de software. \\ \hline
     RAS[1] e RAS[2] 
     & A arquitetura de referência de possibilitar que repositórios de  ativos de software desenvolvidos para persistir diferentes tipos  de ativos possam ser facilmente integrados.

A arquitetura de referência deve possibilitar que repositórios de ativos de software implementados em linguagens de  programação distintas e sob diferentes plataformas possam ser  facilmente integrados.
    &  A biblioteca de controle de ativos de software deve fornecer mecanismos de integração que permitem a persistencia de diferentes tipos de ativos implementados em diferentes linguagens de programação.
 \\ \hline
 
  RAS[3] 
  & A arquitetura de referência deve prover mecanismos para que  repositórios de ativos de software na forma de serviços possam  ser publicados e posteriormente descobertos por aplicações  cliente.
  & A biblioteca de controle de ativos deve prover mecanismos para que suas funcionalidades sejam executadas na forma de serviços, que serão publicados e posteriormente descobertos por aplicações clientes.

 \\ \hline
  RAS[4] & 
A arquitetura de referência de prover mecanismos para que  repositórios de ativos de software orientados a serviço possam  ser compostos por processos de negócio ou utilizados por  aplicações cliente. & Requisitos não-funcionais 1: A biblioteca de controle de ativos de software deve permitir acesso externo de maneira automatizada
2: A biblioteca de controle de ativos de software deve permitir que serviços sejam usados por meio de orquestração (Camada de webservice permitirá isso).


 \\ \hline
  RAS[5] & 
A arquitetura de referência deve viabilizar o desenvolvimento  de repositórios de ativos de software que disponibilizem  informações sobre suas características e direções normativas de  uso, por meio de descrições padronizadas.
 & A biblioteca de controle 
deve garantir que o desenvolvimento de repositórios de ativos informem suas características  e direções normativas de uso por meio de descrições padronizadas de suas funcionalidades.

 \\ \hline
RAS[6] & 
A arquitetura de referência deve viabilizar o desenvolvimento  de repositório de ativos de software que disponibilizem  descrições semânticas, permitindo assim sua classificação nos  repositórios de serviço. & O escopo inicial do desenvolvimento da biblioteca hidra tem como foco os requisitos fundamentais de repositorio de ativos de software, transportando o requisito RAS[6] para uma abordagem futura em uma nova análise de escopo
\\ \hline
RAS[7] & 
A arquitetura de referência deve viabilizar o desenvolvimento  de repositório de ativos de software que tenham à disposição  informações e documentos relacionados às suas características  de qualidade. & O escopo inicial do desenvolvimento da biblioteca hidra tem como foco os requisitos fundamentais de repositorio de ativos de software, transportando o requisito RAS[7] para uma abordagem futura em uma nova análise de escopo.
 \\ \hline 

RAS[8] & 
A arquitetura de referência deve prover mecanismos para a  captura, monitoramento, registro e sinalização do não  cumprimento de requisitos de qualidade estabelecidos entre  serviços provedores e serviços clientes. & O escopo inicial do desenvolvimento da biblioteca hidra tem como foco os requisitos fundamentais de repositorio de ativos de software, transportando o requisito RAS[8] para uma abordagem futura em uma nova análise de escopo.
 \\ \hline 

RAS[9] & 
A arquitetura de referência deve viabilizar o desenvolvimento
de repositório de ativos de software escalável, capaz de evoluir 
de maneira incremental, por meio da composição de novas 
funcionalidades disponíveis na forma de serviços. & A biblioteca de controle deve prover mecanismos a fim de permitir a adição de novas funcionalidades a biblioteca de controle, por meio de serviços de serivços.
 \\ \hline 
 RAS[10] & 
A arquitetura de referência deve possibilitar que serviços de  repositório de ativos de software e composições desses  serviços sejam tratados uniformemente, ou seja, possam ser  publicados, localizados e utilizados da mesma forma. & A biblioteca de serviços deve prover mecanismos que permitam a seus serviços serem publicados, localizados e utilizados da mesma forma.
 \\ \hline 
 
 RAS[11] & 

A arquitetura de referência deve possibilitar que serviços do  repositório de ativos de software possam interagir diretamente  ou por meio do uso de barramentos de serviço. & 
A biblioteca de controle deve ter uma camada de abstração que permita a integração entre aplicativos, ou que se comuniquem diretamente.

 \\ \hline 

\end{longtable}

\chapter{Conclusão} \label{chapter:conclusion}

\section{Contribuições}

A principal contribuição deste trabalho é a biblioteca \textit{Hidra}, criada com o apoio da arquitetura de referência \textit{Cambuci}\cite{dissertacaoOsshiro2014}, e que possibilitará a criação de repositórios de ativos de software, podendo futuramente ser extendida. Assim como resaltado em (OSSHIRO, 2014) essa arquitetura de referência contribui efetivamente com a área de reúso de software, pois pode ser especializada para qualquer sub-domínio do domínio de repositórios de ativos de software, como repositórios de LPN, repositórios de requisitos, repositórios de modelos de análise, repositórios de teste de software, entre outros. Adicionalmente também contribuirá com a abordagem GLPN (Landre, 2012), visto que possibilitará a construção de um repositório de LPN.

\section{Trabalhos Futuros}


A biblioteca \textit{Hidra} esta em sua versão inicial, e não aborda todos os aspectos de granularidade de um ativo de software reusável, possibilitando ainda o desenvolvimento de trabalhos futuros dos quais listamos alguns:

\begin{itemize}

\item Extensão da biblioteca permitindo a definição e implementação dos Perfils de um ativo de Software de acordo com o modelo proposto pela \textit{(OMG)}.

\item Extensão da biblioteca de modo a comtemplar especificidades relacionadas a LPNs.

\item Extensão da biblioteca de modo a fornecer mecanismos de buscas de ativos utilizando algoritmos de busca.

\item Extensão da camada de serviços da biblioteca de modo a forncer informações sobre suas caracteristicas normativas de uso, por meio de descrições padronizadas seguindo o padrão DNS para descoberta de serviços.

\item Extensão da camada de serviços da biblioteca de modo a viabilizar as descrições semânticas de um repositório, permitindo assim a sua classificação.

\item Extensão da biblioteca hidra de modo a viabilizar o desenvolvimento de repositórios que tenham a disposição informaões e documentos relacionados às suas caracteristicas de qualidade.

\item Extensão da biblioteca hidra de modo a prover mecanismos para a captura, monitoramento, registro e sinalização do não cumprimento de requisitos de qualidade estabelecidos entre serviços provedores e serviços clientes.

\end{itemize}

%\setglossarystyle{altlisthypergroup}


\bibliographystyle{plain}
\bibliography{referencias}
\newpage




\end{document}
