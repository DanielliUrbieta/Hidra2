\chapter{Hidra} \label{chapter:hidra}

O controle e versionamento de componentes e artefatos de software, é uma das importantes etapas da engenharia de software, e o desenvolvimento em equipes necessita de um gerenciamento de ativos capaz de atender a demanda dos processos ligados ao desenvolvimento de software.

O projeto \textit{Hidra} consiste de uma biblioteca para apoiar o desenvolvimento de repositórios de ativos de software padronizados. Baseada nos conceitos e funcionalidades do famoso e bem sucedido controlador de versões GIT, a biblioteca provê mecanismos úteis para o armazenamento de informações relevantes para a gestão de ativos, e favorece a utilização de suas funcionalidades por meio de serviços integrados. De modo a garantir a qualidade da análise e desenvolvimento deste projeto, as especificações da biblioteca \textit{Hidra} serão apresentada na forma de documentação técnica e documentação funcional.

Neste capitulo são apresentados os principais componentes para o desenvolvimento da biblioteca tex\textit{Hidra}. Na seção 4.1 é apresentada a derivação dos requisitos da arquitetura de referência. Na seção 4.2 são apresentados os Requisitos Funcionais e Não Funcionais. A seção 4.3 é composta pelo glossário. A seção 4.4 apresenta o diagrama de componentes ao qual representa o projeto da biblioteca tex\textit{Hidra}. Por fim a seção 4.5 apresenta o diagrama de classes que compõe o projeto.

\section{Derivação dos Requisitos da Biblioteca Hidra a partir dos Requisios Arquiteturais da Cambuci}


Os requisitos funcionais e não-funcionais da biblioteca \textit{Hidra} foram elicitados a partir da derivações dos requisitos arquiteturais da arquitetura de referência Cambuci \cite{dissertacaoOsshiro2014}, bem como, a partir da identificação de novos requisitos voltados diretamente para a definição da biblioteca \textit{Hidra}.

A tabela a seguir apresenta os requisitos arquiteturais da arquitetura de referência Cambuci (colunas ID e Requisito Original), juntamente com os requisitos respectivamente derivados à biblioteca \textit{Hidra} (coluna Requisitos Derivados). Os requisitos específicos da biblioteca \textit{Hidra} foram identificados adotando como padrão as siglas RF para os requisitos funcionais e RN para a requisitos não-funcionais.

\newpage

\begin{longtable}{ | l | p{9cm} | p{3cm} |}
\caption{Tabela de Requisitos Hidra}\\
\hline
\textbf{ID} & \textbf{Requisito Original} & \textbf{Requisito Derivado}  \\
\hline
\endfirsthead
\multicolumn{3}{c}%
{\tablename\ \thetable\ -- \textit{Tabela de Requisitos Arquiteturais Cambuci}} \\
\hline
\textbf{ID} & \textbf{Requisito Original} & \textbf{Requisito Derivado}  \\
\hline
\endhead
\hline \multicolumn{3}{r}{\textit{Continua na página seguinte}} \\
\endfoot
\hline
\endlastfoot
	RA-AS[1]
	& A arquitetura de referência deve possibilitar que repositórios de ativos de software incluam um novo ativo, que pode ser composto por vários artefatos.
	& RF-01 e RF-02 \\ \hline
    
    RA-AS[2] 
    & A arquitetura de referência deve possibilitar que repositórios de ativos de software forneçam mecanismo para aceitação e certificação de ativos.
    & RF-03 e RF-04 \\ \hline

    RA-AS[3]
    & A arquitetura de referência deve possibilitar que repositórios de ativos de software desativem ativos que não serão mais utilizados.
    & RF-05 \\ \hline
     
    RA-AS[4] 
    & A arquitetura de referência deve possibilitar que repositórios de ativos de software permitam a classificação de um ativo e também informar o contexto de sua utilização.
    & RF-06 e RF-07 \\ \hline

	RA-AS[5] 
	& A arquitetura de referência deve possibilitar que repositórios de ativos de software registrem a dependência entre ativos.
	& RF-08 \\ \hline

    RA-AS[6] 
    & A arquitetura de referência deve possibilitar que repositórios de ativos de software notifiquem os interessados sobre mudanças que aconteçam no ativo. 
    & RF-09
 	\\ \hline
 
    RA-AS[7] 
    & A arquitetura de referência deve possibilitar que  repositórios de ativos de software permitam realizar  buscas e recuperação dos ativos 
    & RF-10, RF-11, RF-12
    \\ \hline
    
    \underline{RA-AS[8]}
    & A arquitetura de referência deve possibilitar que  repositórios de ativos de software permitam a  navegação entre ativos 
    & Não derivado para a versão atual da \textit{Hidra} (Trabalho Futuro). 
    \\ \hline
    
    RA-AS[9] 
    & A arquitetura de referência deve possibilitar que  repositórios de ativos de software aceite múltiplas  fontes de origem de ativos, com o objetivo de facilitar  a integração entre equipes e entre repositórios  diferentes.  
    & RN-01
    \\ \hline
 
    RA-AS[10]
    & A arquitetura de referência deve possibilitar que  repositórios de ativos de software criem e armazenem  múltiplas versões de um mesmo ativo.
    & RN-02
    \\ \hline

    RA-AS[11] 
    & A arquitetura de referência deve possibilitar que  repositórios de ativos de software gerencie a  configuração, como por exemplo, a definição dos itens  do ativo que são configuráveis, o controle de  mudanças dos itens do ativo que são configuráveis.
    & RN-03
    \\ \hline
    
    \underline{RA-AS[12]}
    &A arquitetura de referência deve possibilitar que  repositórios de ativos de software permita o registro de  impressões dos usuários a respeito da versão do ativo  que eles utilizaram. 
    & Não derivado para a versão atual da \textit{Hidra} (Trabalho Futuro).
    \\ \hline

    \underline{RA-AS[13]}
    & A arquitetura de referência deve possibilitar que  repositórios de ativos de software registrem métricas  coletadas sobre a utilização do ativo.
    & Fora do escopo da \textit{Hidra} (Trabalho Futuro).
    \\ \hline

    \underline{RA-AS[14]}
    & A arquitetura de referência deve possibilitar que  repositórios de ativos de software ofereçam  informações relativas ao reúso, iniciativas de reúso,  ativos mais usados, etc.
    & Fora do escopo da \textit{Hidra} (Trabalho Futuro). 
    \\ \hline
    
    \underline{RA-AS[15]}
    & A arquitetura de referência deve possibilitar que  repositórios de ativos de software permitam o acesso de acordo com o papel que o usuário assume.
    & Não derivado para a versão atual da \textit{Hidra} (Trabalho Futuro). 
    \\ \hline

    RA-AS[16] 
    & A arquitetura de referência deve possibilitar que  repositórios de ativos de software garantam a  integridade dos ativos, ou seja, que eles não sofram  alterações não autorizadas.
    & RN-04
    \\ \hline

    RA-AS[17] 
    & A arquitetura de referência deve possibilitar que  repositórios de ativos de software realizem o  gerenciamento de transação, garantindo a atomicidade,  consistência, isolamento e durabilidade.
    & RF-13 
    \\ \hline

    RAS[1]
    & A arquitetura de referência de possibilitar que repositórios de  ativos de software desenvolvidos para persistir diferentes tipos  de ativos possam ser facilmente integrados.
    & RF-14
    \\ \hline

    RAS[2] 
    & A arquitetura de referência deve possibilitar que repositórios de ativos de software implementados em linguagens de  programação distintas e sob diferentes plataformas possam ser  facilmente integrados.
    & RF-15 
    \\ \hline

    RAS[3] 
    & A arquitetura de referência deve prover mecanismos para que  repositórios de ativos de software na forma de serviços possam  ser publicados e posteriormente descobertos por aplicações  cliente.
    & RN-01 
    \\ \hline
    
    RAS[4] & 
    A arquitetura de referência de prover mecanismos para que  repositórios de ativos de software orientados a serviço possam  ser compostos por processos de negócio ou utilizados por  aplicações cliente. 
    & RN-01
    
    e RN-04 
    \\ \hline

    RAS[5] & 
    A arquitetura de referência deve viabilizar o desenvolvimento  de repositórios de ativos de software que disponibilizem  informações sobre suas características e direções normativas de  uso, por meio de descrições padronizadas.
    & RF-16
    \\ \hline

    RAS[6] 
    & A arquitetura de referência deve viabilizar o desenvolvimento  de repositório de ativos de software que disponibilizem  descrições semânticas, permitindo assim sua classificação nos  repositórios de serviço.
    & RF-17
    \\ \hline

    RAS[7] 
    & A arquitetura de referência deve viabilizar o desenvolvimento  de repositório de ativos de software que tenham à disposição  informações e documentos relacionados às suas características  de qualidade. 
    & RF-18
    \\ \hline 

    RAS[8] 
    & A arquitetura de referência deve prover mecanismos para a  captura, monitoramento, registro e sinalização do não  cumprimento de requisitos de qualidade estabelecidos entre  serviços provedores e serviços clientes. 
    & RF-19
    \\ \hline 
    RAS[9] 
	& A arquitetura de referência deve viabilizar o desenvolvimento
	de repositório de ativos de software escalável, capaz de evoluir 
	de maneira incremental, por meio da composição de novas 
	funcionalidades disponíveis na forma de serviços. 
	& RN-05
	\\ \hline 

	RAS[10] 
	& A arquitetura de referência deve possibilitar que serviços de  repositório de ativos de software e composições desses  serviços sejam tratados uniformemente, ou seja, possam ser  publicados, localizados e utilizados da mesma forma. 
	& RN-01
	\\ \hline 

	RAS[11] 
	& A arquitetura de referência deve possibilitar que serviços do  repositório de ativos de software possam interagir diretamente  ou por meio do uso de barramentos de serviço. 
	& RN-01
	\\ \hline 

\end{longtable}

\section{Requisitos Funcionais Hidra}

A partir da Tabela 4.1 os seguintes requisitos funcionais da biblioteca Hidra foram criados.

\begin{longtable}{ | l | p{6cm} | p{2cm} | p{4cm} |}
\caption{Requisitos Funcionais Hidra}\\
\hline
\textbf{ID} & \textbf{Requisito} & \textbf{Cambuci} & \textbf{Solução}  \\
\hline
\endfirsthead
\multicolumn{4}{c}%
{\tablename\ \thetable\ -- \textit{Requisitos Funcionais Hidra}} \\
\hline
\textbf{ID} & \textbf{Requisito} & \textbf{RA Cambuci} & \textbf{Solução}  \\
\hline
\endhead
\hline \multicolumn{4}{r}{\textit{Continua na página seguinte}} \\
\endfoot
\hline
\endlastfoot
	RF-01
	& A biblioteca \textit{Hidra} deve possibilitar a \textbf{inclusão de ativos de software}, levando em consideração a composição de um ativo por diferentes artefatos.
	& RA-AS[1]
	& Os ativos reusáveis de software são armazenados no repositório em forma de diretórios, por meio, da segunda forma de armazenamento RAS \cite{omg2005}. \\ \hline

	% 4_1_rf_hidra.tex
	RF-02
	& A biblioteca \textit{Hidra} deve fornecer mecanismos a fim de \textbf{listar artefatos que compõem um ativo de software} armazenado no repositório.
	& RA-AS[1]
	& Requisito implementado por meio dos métodos \textit{Asset.getSolution()} e \textit{Asset.setSolution()}. \\ \hline

	RF-03
	& A biblioteca \textit{Hidra} deve possuir uma \textbf{estrutura padronizada de representação, comunicação e armazenamento} de ativos de software. 
	& RA-AS[2] 
	& Foi adotado o padrão RAS atualmente em sua versão 2.2. \\ \hline

	RF-04
	& A biblioteca \textit{Hidra} deve possibilitar que todo novo ativo de software seja \textbf{validado e certificado} de acordo com o padrão adotado.
	& RA-AS[2]
	& As regras especificadas no padrão RAS, expressas em forma de um XSD \textit{NomeArquivoXSD.xsd}, são consultadas ao validar e certificar um Ativo antes de qualquer atualização ou inserção (método \textit{Asset.validate()}).\\ \hline

	RF-05
	& A biblioteca \textit{Hidra} deve possibilitar que ativos de software, que não forem mais utilizados, \textbf{sejam removidos do repositório}.
	& RA-AS[3]
	& Requisito implementado por meio do método \textit{Repository .removeAsset(Asset asset)}. \\ \hline

	RF-06
	& A biblioteca \textit{Hidra} deve possibilitar a adição de informações para \textbf{classificação de um ativo} e também o contexto de sua utilização.
	& RA-AS[4] 
	& Requisito implementado por meio dos métodos \textit{Asset.getClassification()} e \textit{Asset.setClassification()}. \\ \hline

	RF-07
	& A biblioteca \textit{Hidra} deve possibilitar a adição de informações sobre \textbf{regras para instalação, personalização, e utilização} do ativo.
	& extensão do requisito RA-AS[4] baseando-se no padrão RAS.
	& Requisito implementado por meio dos métodos \textit{Asset.getUsage()} e \textit{Asset.setUsage()}. \\ \hline

	RF-08
	& A biblioteca \textit{Hidra} deve possibilitar o \textbf{registro de dependência entre ativos}.
	& RA-AS[5] 
	& Requistio implementado por meio dos métodos \textit{Asset.getRelatedAssets()} e \textit{Asset.setRelatedAsset()}. \\ \hline

	RF-09
	& A biblioteca \textit{Hidra} deve fornecer mecanismos a fim de \textbf{oferecer informações relevantes a todos os interessados, sobre mudanças} que aconteçam no ativo de software: data de alteração, autor da alteração, o que foi alterado e descrição sobre a alteração.
	& RA-AS[6] 
	& Requisito implementado por meio do método \textit{Asset.getLog()} \\ \hline

	RF-10
	& A biblioteca \textit{Hidra} deve fornecer mecanismos a fim de \textbf{listar ativos armazenados no repositório}.
	& RA-AS[7]
	& Requisito implementado por meio do método \textit{Repository.listAssets()} \\ \hline

	RF-11
	& A biblioteca \textit{Hidra} deve fornecer mecanismos a fim de \textbf{recuperar um ativo armazenado no repositório} (\textit{download}).
	& RA-AS[7] 
	& Requisito implementado por meio do método \textit{Repository .retrieveAsset()} \\ \hline

	\underline{RF-12}
	& A biblioteca \textit{Hidra} deve fornecer mecanismos a fim de \textbf{buscar ativos armazenados no repositório}.
	& RA-AS[7] 
	& Requisito não implementado na versão atual da biblioteca \textit{Hidra} (Trabalho Futuro). \\ \hline

	RF-13
	& A biblioteca \textit{Hidra} deve fornecer mecanismos a fim de \textbf{garantir a atomicidade, consistência e isolamento de transações de controle} de ativos de software
	& RA-AS[17]
	& A biblioteca provê recursos para que as transações sejam controladas considerando os aspectos citados: os recursos da API jGit e a camada de serviços. Mas essa implementação deverá ser realizada diretamente no repositório. \\ \hline

	RF-14
	& A biblioteca \textit{Hidra} deve fornecer mecanismos a fim de \textbf{permitir a persistência de diferentes tipos de ativos}.
	& RAS[1]
	& O padrão RAS, adotado na biblioteca \textit{Hidra} para a implementação dos requisitos relacionados ao Ativo de Software, permite a persistência de diferentes tipos de ativos, desde que a estrutura de cada ativo seja descrita em sua solução (métodos \textit{Asset.getSolution()} e \textit{Asset.setSolution()}). \\ \hline

	RF-15
	& A biblioteca \textit{Hidra} deve fornecer mecanismos a fim de \textbf{permitir a persistência de ativos implementados em diferentes linguagens de programação}.
	& RAS[2] 
	& O padrão RAS, adotado na biblioteca \textit{Hidra} para a implementação dos requisitos relacionados ao Ativo de Software, permite a persistência de ativos implementados em diferentes linguagens de programação, desde que as regras para instalação, personalização, e utilização de cada ativo seja descrita na especificação de seu uso (métodos \textit{Asset.getUsage()} e \textit{Asset.setUsage()}). \\ \hline

	RF-16
	& A biblioteca \textit{Hidra} deve fornecer uma \textbf{camada de serviços (\textit{Webservice})} com informações sobre suas características e direções normativas de uso, por meio de descrições padronizadas seguindo o padrão DNS para descoberta de serviços.
	& RAS[5]
	& Requisito não implementado na versão atual da biblioteca \textit{Hidra} (Trabalho Futuro). \\ \hline

	RF-17
	& A biblioteca \textit{Hidra} deve viabilizar o desenvolvimento de um repositório de ativos de software com uma camada \textit{webservice} com descrições semânticas, permitindo assim sua classificação nos respositórios de serviços.
	& RAS[6]
	& Requisito não implementado na versão atual da biblioteca \textit{Hidra} (Trabalho Futuro). \\ \hline

	RF-18
	& A biblioteca \textit{Hidra} deve viabilizar o desenvolvimento  de repositório de ativos de software que tenham à disposição  informações e documentos relacionados às suas características  de qualidade.
	& RAS[7]
	& Requisito não implementado na versão atual da biblioteca \textit{Hidra} (Trabalho Futuro). \\ \hline

	RF-19
	& A biblioteca \textit{Hidra} deve prover mecanismos para a  captura, monitoramento, registro e sinalização do não  cumprimento de requisitos de qualidade estabelecidos entre  serviços provedores e serviços clientes.
	& RAS[8]
	& Requisito não implementado na versão atual da biblioteca \textit{Hidra} (Trabalho Futuro). \\ \hline


\end{longtable}

\begin{longtable}{ | l | p{4cm} | p{2cm} | p{6cm} |}
\caption{Requisitos Não-Funcionais Hidra}\\
\hline
\textbf{ID} & \textbf{Requisito} & \textbf{RA Cambuci} & \textbf{Solução}  \\
\hline
\endfirsthead
\multicolumn{4}{c}%
{\tablename\ \thetable\ -- \textit{Requisitos Não-funcionais Hidra}} \\
\hline
\textbf{ID} & \textbf{Requisito} & \textbf{RA Cambuci} & \textbf{Solução}  \\
\hline
\endhead
\hline \multicolumn{4}{r}{\textit{Continua na página seguinte}} \\
\endfoot
\hline
\endlastfoot
	RN-01
	& A biblioteca \textit{Hidra} deve fornecer mecanismos para que repositórios de ativos de software aceitem múltiplas fontes de origem de ativos, por meio de serviços web que possam ser publicados, localizados e utilizados de maneira uniforme.
	& RA-AS[9],

	RAS[3],

	RAS[4], 

	RAS[10]
	& Requisito é atendido por meio da camada de serviço provida pela biblioteca, que segue o padrão REST, que permitirá ao repositório desenvolvido, tendo como base a \textit{Hidra}, fácil acesso e integração a múltiplas ferramentas.
	\\ \hline

	RN-02
	& A biblioteca \textit{Hidra} deve fornecer mecanismos de versionamento aos ativos de software.
	& RA-AS[10]
	& Requisito é atendido por meio da camada de persistência provida pela biblioteca, que utiliza a API jGit para manipulação das operações de Gerenciamento de Configuração sobre um repositório Git, e permitirá que o repositório armazene múltiplas versões de um mesmo ativo.
	\\ \hline

	RN-03
	& A biblioteca \textit{Hidra} deve oferecer mecanismos  para  gerenciamento da configuração de ativos de software.
	& RA-AS[11] 
	& Requisito é atendido por meio da camada de persistência provida pela biblioteca, que utiliza a API jGit para manipulação das operações de Gerenciamento de Configuração sobre um repositório Git, e permitirá que o repositório gerencie configurações de ativos de software.
	\\ \hline

	RN-04
	& A biblioteca \textit{Hidra} permitir que repositórios garantam que seus ativos de software não sofram alterações não autorizadas.
	& RA-AS[16],

	RAS[4]
	& Requisito é atendido por meio do controle de usuários da camada de persistência provida pela biblioteca, que utiliza a API jGit para manipulação das operações de Gerenciamento de Configuração sobre um repositório Git. Na versão inicial, a biblioteca utiliza um usuário padrão informado no seu arquivo de propriedades (hidra.properties).
	\\ \hline

	RN-05
	& A biblioteca \textit{Hidra} deve ser extensível de modo a viabilizar o desenvolvimento de repositório de ativos de software escalável, capaz de evoluir de maneira incremental, por meio da composição de novas funcionalidades disponíveis na forma de serviços.
	& RAS[9] 
	& Requisito é atendido por meio dos padrões adotados para a implementação da biblioteca: i) Padrão RAS para representação e manipulação de Ativos de Software Reusáveis; ii) Divisão dos recursos providos em camadas (jGit para persistência, Hidra para regras de negócio, HidraService para fornecimento de serviços; iii) Padrões de Projeto (tanto padrões GRASP quanto padrões GoF) adotados na implementação, como por exemplo, Singleton, Facade, Strategy, Especialista na Informação).
 \\ \hline 

\end{longtable}




\glossary{name={Arquitetura de referência}, description={A estrutura e organização lógica de funcionamento do sistema computacional.}}

\glossary{name={Artefato}, description={Qualquer produto que pode ser criado, armazenado e manipulado por produtores / consumidores ou por uma ferramenta, dentro de um processo de desenvolvimento de software.}}

\glossary{name={Ativo}, description={É o elemento principal do RAS; descreve e empacota os artefatos e provê a documentação sobre como e qual problema ele resolve.}}

\glossary{name={Descritor(es)}, description={Descreve  os comportamentos e as caracteristicas do ativo.}}

\glossary{name={Repositório}, description={Representa o conceito de um armazém de ativos, que é responsável por prover um conjunto de mecanismos para a gestão e uso dos ativos}}

\glossary{name={Versionamento}, description={Criação e armazenamento de multiplas versões de um mesmo ativo.}}

\glossary{name={Web Services}, description={Recursos automatizados acessíveis pela Internet. São recursos de software ou componentes funcionais com capacidades que podem ser acessadas através de um endereço universal de Internet. Web services costumam usar XML para interagir com outros sistemas}}


\glossary{name={OMG}, description={Organização internacional que aprova padrões abertos para aplicações orientadas a objetos. Esse grupo define também a OMA (Object Management Architecture), um modelo padrão de objeto para ambientes distribuídos.}}

\glossary{name={RAS}, description={Especificação de Ativos Reusáveis}}






