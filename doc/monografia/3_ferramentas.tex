\chapter{Tecnologias e Ferramentas} \label{chapter:ferramentas}

\section{Descrição das tecnologias e ferramentas utilizadas}
Esta seção apresenta as ferramentas e tecnologias utilizadas no desenvolvimento do projeto \textit{Hidra}.



\begin{itemize}
\item \textbf{NetBeans IDE (Integrated Development Environment)}. Ambiente de desenvolvimento utilizado. Ferramenta para que permite escrever, compilar, depurar e instalar programas. O IDE é completamente escrito em Java, mas também da suporte a qualquer linguagem de programação. possui um grande número de módulos para extender suas funcionalidades. O NetBeans IDE é livre, sem restrições à sua forma de utilização.

\item \textbf{JDK 8 (Java Development Kit)}. Inclui ferramentas úteis para desenvolver e testar os programas escritos na linguagem de programação Java e em execução na plataforma Java.

\item \textbf{JAXB (Java Architecture for XML Binding)}. \cite{dissertacaoHenriqueFaria2005} Auxilia na manipulação de documentos XML, através da linguagem de programação Java. Ao invés de percorrer o documento, separá-lo logicamente em partes discretas, e repassar o conteúdo para a aplicação Java (ORT, 2003), assim como o SAX (SAX, 2005) ou o DOM (W3C, 2005), o JAXB funciona em duas etapas.
    \begin{itemize}
    \item Um esquema XML é compilado em códigos fonte de classes.
    
    \item Após o mapeamento (\textit{binding}) de um esquema XML em um conjunto de classes, então, pode-se, através de uma aplicação que utilize as bibliotecas do JAXB e as classes geradas pelo compilador:
        \begin{itemize}

        \item criar uma árvore de objetos em memória a partir de um documento XML (\textit{unmarshalling});
        
        \item criar um documento XML a partir de objetos Java (\textit{marshalling});
        
        \end{itemize} 
    \end{itemize}
    

\item \textbf{JGit}. Biblioteca Java do sistema de controle de versão Git, na qual contém.
    \begin{itemize}
    \item rotinas de acesso ao repositório
    
    \item protocolos de rede
    
    \item principais algoritmos de controle de versão
    \end{itemize}

\item \textbf{Apache Maven}. Ferramenta de automação para compilação utilizada em projetos Java,  Similar à ferramenta Ant, hospedada pela Apache Software Foundation. Utiliza-se de um arquivo XML (\textit{POM}) para descrever o projeto de software sendo construído, suas dependências sobre módulos e componentes externos, a ordem de compilação, diretórios e plug-ins necessários. O maven Importa bibliotecas Java e seus plug-ins dinamicamente de um ou mais repositórios online, como o \textit{Maven 2 Central Repository}.


\item \textbf{java.util.Properties}. Classe Java que permite armazenar informações indexadas por uma palavra chave. A partir dela é possível fazer uma rápida pesquisa para recuperar informações necessárias em aplicativos Java SE e Java EE. As informações armazenadas por esta classe podem permanecer em memória ou ser persistida em um arquivo texto.

\item \textbf{Padrões de Projeto Grasp e Grop}. \cite{larman2004} Utilização de padrões Facade e Especialista na Informação.

\textbf{Facade}: Classe Facade para simplificar o acesso a subconjuntos de métodos e classes, diminuição do acoplamento, e melhor manutenibilidade.

\textbf{Especialista na Informação}: Padrão utilizado para atribuição de responsabilidades. As responsabilidades serão atribuidas a quem realmente detêm a informação necessária para preencher os requisitos daquela responsabilidade.

\end{itemize}