\section{Requisitos Funcionais Hidra}

A partir da Tabela 4.1 os seguintes requisitos funcionais da biblioteca Hidra foram criados.

\begin{longtable}{ | l | p{6cm} | p{2cm} | p{4cm} |}
\caption{Requisitos Funcionais Hidra}\\
\hline
\textbf{ID} & \textbf{Requisito} & \textbf{Cambuci} & \textbf{Solução}  \\
\hline
\endfirsthead
\multicolumn{4}{c}%
{\tablename\ \thetable\ -- \textit{Requisitos Funcionais Hidra}} \\
\hline
\textbf{ID} & \textbf{Requisito} & \textbf{RA Cambuci} & \textbf{Solução}  \\
\hline
\endhead
\hline \multicolumn{4}{r}{\textit{Continua na página seguinte}} \\
\endfoot
\hline
\endlastfoot
	RF-01
	& A biblioteca \textit{Hidra} deve possibilitar a \textbf{inclusão de ativos de software}, levando em consideração a composição de um ativo por diferentes artefatos.
	& RA-AS[1]
	& Os ativos reusáveis de software são armazenados no repositório em forma de diretórios, por meio, da segunda forma de armazenamento RAS \cite{omg2005}. \\ \hline

	% 4_1_rf_hidra.tex
	RF-02
	& A biblioteca \textit{Hidra} deve fornecer mecanismos a fim de \textbf{listar artefatos que compõem um ativo de software} armazenado no repositório.
	& RA-AS[1]
	& Requisito implementado por meio dos métodos \textit{Asset.getSolution()} e \textit{Asset.setSolution()}. \\ \hline

	RF-03
	& A biblioteca \textit{Hidra} deve possuir uma \textbf{estrutura padronizada de representação, comunicação e armazenamento} de ativos de software. 
	& RA-AS[2] 
	& Foi adotado o padrão RAS atualmente em sua versão 2.2. \\ \hline

	RF-04
	& A biblioteca \textit{Hidra} deve possibilitar que todo novo ativo de software seja \textbf{validado e certificado} de acordo com o padrão adotado.
	& RA-AS[2]
	& As regras especificadas no padrão RAS, expressas em forma de um XSD \textit{NomeArquivoXSD.xsd}, são consultadas ao validar e certificar um Ativo antes de qualquer atualização ou inserção (método \textit{Asset.validate()}).\\ \hline

	RF-05
	& A biblioteca \textit{Hidra} deve possibilitar que ativos de software, que não forem mais utilizados, \textbf{sejam removidos do repositório}.
	& RA-AS[3]
	& Requisito implementado por meio do método \textit{Repository .removeAsset(Asset asset)}. \\ \hline

	RF-06
	& A biblioteca \textit{Hidra} deve possibilitar a adição de informações para \textbf{classificação de um ativo} e também o contexto de sua utilização.
	& RA-AS[4] 
	& Requisito implementado por meio dos métodos \textit{Asset.getClassification()} e \textit{Asset.setClassification()}. \\ \hline

	RF-07
	& A biblioteca \textit{Hidra} deve possibilitar a adição de informações sobre \textbf{regras para instalação, personalização, e utilização} do ativo.
	& extensão do requisito RA-AS[4] baseando-se no padrão RAS.
	& Requisito implementado por meio dos métodos \textit{Asset.getUsage()} e \textit{Asset.setUsage()}. \\ \hline

	RF-08
	& A biblioteca \textit{Hidra} deve possibilitar o \textbf{registro de dependência entre ativos}.
	& RA-AS[5] 
	& Requistio implementado por meio dos métodos \textit{Asset.getRelatedAssets()} e \textit{Asset.setRelatedAsset()}. \\ \hline

	RF-09
	& A biblioteca \textit{Hidra} deve fornecer mecanismos a fim de \textbf{oferecer informações relevantes a todos os interessados, sobre mudanças} que aconteçam no ativo de software: data de alteração, autor da alteração, o que foi alterado e descrição sobre a alteração.
	& RA-AS[6] 
	& Requisito implementado por meio do método \textit{Asset.getLog()} \\ \hline

	RF-10
	& A biblioteca \textit{Hidra} deve fornecer mecanismos a fim de \textbf{listar ativos armazenados no repositório}.
	& RA-AS[7]
	& Requisito implementado por meio do método \textit{Repository.listAssets()} \\ \hline

	RF-11
	& A biblioteca \textit{Hidra} deve fornecer mecanismos a fim de \textbf{recuperar um ativo armazenado no repositório} (\textit{download}).
	& RA-AS[7] 
	& Requisito implementado por meio do método \textit{Repository .retrieveAsset()} \\ \hline

	\underline{RF-12}
	& A biblioteca \textit{Hidra} deve fornecer mecanismos a fim de \textbf{buscar ativos armazenados no repositório}.
	& RA-AS[7] 
	& Requisito não implementado na versão atual da biblioteca \textit{Hidra} (Trabalho Futuro). \\ \hline

	RF-13
	& A biblioteca \textit{Hidra} deve fornecer mecanismos a fim de \textbf{garantir a atomicidade, consistência e isolamento de transações de controle} de ativos de software
	& RA-AS[17]
	& A biblioteca provê recursos para que as transações sejam controladas considerando os aspectos citados: os recursos da API jGit e a camada de serviços. Mas essa implementação deverá ser realizada diretamente no repositório. \\ \hline

	RF-14
	& A biblioteca \textit{Hidra} deve fornecer mecanismos a fim de \textbf{permitir a persistência de diferentes tipos de ativos}.
	& RAS[1]
	& O padrão RAS, adotado na biblioteca \textit{Hidra} para a implementação dos requisitos relacionados ao Ativo de Software, permite a persistência de diferentes tipos de ativos, desde que a estrutura de cada ativo seja descrita em sua solução (métodos \textit{Asset.getSolution()} e \textit{Asset.setSolution()}). \\ \hline

	RF-15
	& A biblioteca \textit{Hidra} deve fornecer mecanismos a fim de \textbf{permitir a persistência de ativos implementados em diferentes linguagens de programação}.
	& RAS[2] 
	& O padrão RAS, adotado na biblioteca \textit{Hidra} para a implementação dos requisitos relacionados ao Ativo de Software, permite a persistência de ativos implementados em diferentes linguagens de programação, desde que as regras para instalação, personalização, e utilização de cada ativo seja descrita na especificação de seu uso (métodos \textit{Asset.getUsage()} e \textit{Asset.setUsage()}). \\ \hline

	RF-16
	& A biblioteca \textit{Hidra} deve fornecer uma \textbf{camada de serviços (\textit{Webservice})} com informações sobre suas características e direções normativas de uso, por meio de descrições padronizadas seguindo o padrão DNS para descoberta de serviços.
	& RAS[5]
	& Requisito não implementado na versão atual da biblioteca \textit{Hidra} (Trabalho Futuro). \\ \hline

	RF-17
	& A biblioteca \textit{Hidra} deve viabilizar o desenvolvimento de um repositório de ativos de software com uma camada \textit{webservice} com descrições semânticas, permitindo assim sua classificação nos respositórios de serviços.
	& RAS[6]
	& Requisito não implementado na versão atual da biblioteca \textit{Hidra} (Trabalho Futuro). \\ \hline

	RF-18
	& A biblioteca \textit{Hidra} deve viabilizar o desenvolvimento  de repositório de ativos de software que tenham à disposição  informações e documentos relacionados às suas características  de qualidade.
	& RAS[7]
	& Requisito não implementado na versão atual da biblioteca \textit{Hidra} (Trabalho Futuro). \\ \hline

	RF-19
	& A biblioteca \textit{Hidra} deve prover mecanismos para a  captura, monitoramento, registro e sinalização do não  cumprimento de requisitos de qualidade estabelecidos entre  serviços provedores e serviços clientes.
	& RAS[8]
	& Requisito não implementado na versão atual da biblioteca \textit{Hidra} (Trabalho Futuro). \\ \hline


\end{longtable}

\begin{longtable}{ | l | p{4cm} | p{2cm} | p{6cm} |}
\caption{Requisitos Não-Funcionais Hidra}\\
\hline
\textbf{ID} & \textbf{Requisito} & \textbf{RA Cambuci} & \textbf{Solução}  \\
\hline
\endfirsthead
\multicolumn{4}{c}%
{\tablename\ \thetable\ -- \textit{Requisitos Não-funcionais Hidra}} \\
\hline
\textbf{ID} & \textbf{Requisito} & \textbf{RA Cambuci} & \textbf{Solução}  \\
\hline
\endhead
\hline \multicolumn{4}{r}{\textit{Continua na página seguinte}} \\
\endfoot
\hline
\endlastfoot
	RN-01
	& A biblioteca \textit{Hidra} deve fornecer mecanismos para que repositórios de ativos de software aceitem múltiplas fontes de origem de ativos, por meio de serviços web que possam ser publicados, localizados e utilizados de maneira uniforme.
	& RA-AS[9],

	RAS[3],

	RAS[4], 

	RAS[10]
	& Requisito é atendido por meio da camada de serviço provida pela biblioteca, que segue o padrão REST, que permitirá ao repositório desenvolvido, tendo como base a \textit{Hidra}, fácil acesso e integração a múltiplas ferramentas.
	\\ \hline

	RN-02
	& A biblioteca \textit{Hidra} deve fornecer mecanismos de versionamento aos ativos de software.
	& RA-AS[10]
	& Requisito é atendido por meio da camada de persistência provida pela biblioteca, que utiliza a API jGit para manipulação das operações de Gerenciamento de Configuração sobre um repositório Git, e permitirá que o repositório armazene múltiplas versões de um mesmo ativo.
	\\ \hline

	RN-03
	& A biblioteca \textit{Hidra} deve oferecer mecanismos  para  gerenciamento da configuração de ativos de software.
	& RA-AS[11] 
	& Requisito é atendido por meio da camada de persistência provida pela biblioteca, que utiliza a API jGit para manipulação das operações de Gerenciamento de Configuração sobre um repositório Git, e permitirá que o repositório gerencie configurações de ativos de software.
	\\ \hline

	RN-04
	& A biblioteca \textit{Hidra} permitir que repositórios garantam que seus ativos de software não sofram alterações não autorizadas.
	& RA-AS[16],

	RAS[4]
	& Requisito é atendido por meio do controle de usuários da camada de persistência provida pela biblioteca, que utiliza a API jGit para manipulação das operações de Gerenciamento de Configuração sobre um repositório Git. Na versão inicial, a biblioteca utiliza um usuário padrão informado no seu arquivo de propriedades (hidra.properties).
	\\ \hline

	RN-05
	& A biblioteca \textit{Hidra} deve ser extensível de modo a viabilizar o desenvolvimento de repositório de ativos de software escalável, capaz de evoluir de maneira incremental, por meio da composição de novas funcionalidades disponíveis na forma de serviços.
	& RAS[9] 
	& Requisito é atendido por meio dos padrões adotados para a implementação da biblioteca: i) Padrão RAS para representação e manipulação de Ativos de Software Reusáveis; ii) Divisão dos recursos providos em camadas (jGit para persistência, Hidra para regras de negócio, HidraService para fornecimento de serviços; iii) Padrões de Projeto (tanto padrões GRASP quanto padrões GoF) adotados na implementação, como por exemplo, Singleton, Facade, Strategy, Especialista na Informação).
 \\ \hline 

\end{longtable}

