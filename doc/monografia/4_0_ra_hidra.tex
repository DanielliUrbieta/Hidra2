\begin{longtable}{ | l | p{9cm} | p{3cm} |}
\caption{Tabela de Requisitos Hidra}\\
\hline
\textbf{ID} & \textbf{Requisito Original} & \textbf{Requisito Derivado}  \\
\hline
\endfirsthead
\multicolumn{3}{c}%
{\tablename\ \thetable\ -- \textit{Tabela de Requisitos Arquiteturais Cambuci}} \\
\hline
\textbf{ID} & \textbf{Requisito Original} & \textbf{Requisito Derivado}  \\
\hline
\endhead
\hline \multicolumn{3}{r}{\textit{Continua na página seguinte}} \\
\endfoot
\hline
\endlastfoot
	RA-AS[1]
	& A arquitetura de referência deve possibilitar que repositórios de ativos de software incluam um novo ativo, que pode ser composto por vários artefatos.
	& RF-01 e RF-02 \\ \hline
    
    RA-AS[2] 
    & A arquitetura de referência deve possibilitar que repositórios de ativos de software forneçam mecanismo para aceitação e certificação de ativos.
    & RF-03 e RF-04 \\ \hline

    RA-AS[3]
    & A arquitetura de referência deve possibilitar que repositórios de ativos de software desativem ativos que não serão mais utilizados.
    & RF-05 \\ \hline
     
    RA-AS[4] 
    & A arquitetura de referência deve possibilitar que repositórios de ativos de software permitam a classificação de um ativo e também informar o contexto de sua utilização.
    & RF-06 e RF-07 \\ \hline

	RA-AS[5] 
	& A arquitetura de referência deve possibilitar que repositórios de ativos de software registrem a dependência entre ativos.
	& RF-08 \\ \hline

    RA-AS[6] 
    & A arquitetura de referência deve possibilitar que repositórios de ativos de software notifiquem os interessados sobre mudanças que aconteçam no ativo. 
    & RF-09
 	\\ \hline
 
    RA-AS[7] 
    & A arquitetura de referência deve possibilitar que  repositórios de ativos de software permitam realizar  buscas e recuperação dos ativos 
    & RF-10, RF-11, RF-12
    \\ \hline
    
    RA-AS[8] 
    & A arquitetura de referência deve possibilitar que  repositórios de ativos de software permitam a  navegação entre ativos 
    & Não derivado para a versão atual da \textit{Hidra} (Trabalho Futuro). 
    \\ \hline
    
    RA-AS[9] 
    & A arquitetura de referência deve possibilitar que  repositórios de ativos de software aceite múltiplas  fontes de origem de ativos, com o objetivo de facilitar  a integração entre equipes e entre repositórios  diferentes.  
    & RN-01
    \\ \hline
 
    RA-AS[10]
    & A arquitetura de referência deve possibilitar que  repositórios de ativos de software criem e armazenem  múltiplas versões de um mesmo ativo.
    & RN-02
    \\ \hline

    RA-AS[11] 
    & A arquitetura de referência deve possibilitar que  repositórios de ativos de software gerencie a  configuração, como por exemplo, a definição dos itens  do ativo que são configuráveis, o controle de  mudanças dos itens do ativo que são configuráveis.
    & RN-03
    \\ \hline
    
    RA-AS[12] 
    &A arquitetura de referência deve possibilitar que  repositórios de ativos de software permita o registro de  impressões dos usuários a respeito da versão do ativo  que eles utilizaram. 
    & Não derivado para a versão atual da \textit{Hidra} (Trabalho Futuro).
    \\ \hline

    RA-AS[13] 
    & A arquitetura de referência deve possibilitar que  repositórios de ativos de software registrem métricas  coletadas sobre a utilização do ativo.
    & Fora do escopo da \textit{Hidra} (Trabalho Futuro).
    \\ \hline

    RA-AS[14] 
    & A arquitetura de referência deve possibilitar que  repositórios de ativos de software ofereçam  informações relativas ao reúso, iniciativas de reúso,  ativos mais usados, etc.
    & Fora do escopo da \textit{Hidra} (Trabalho Futuro). 
    \\ \hline
    
    RA-AS[15] 
    & A arquitetura de referência deve possibilitar que  repositórios de ativos de software permitam o acesso de acordo com o papel que o usuário assume.
    & Não derivado para a versão atual da \textit{Hidra} (Trabalho Futuro). 
    \\ \hline

    RA-AS[16] 
    & A arquitetura de referência deve possibilitar que  repositórios de ativos de software garantam a  integridade dos ativos, ou seja, que eles não sofram  alterações não autorizadas.
    & RN-04
    \\ \hline

    RA-AS[17] 
    & A arquitetura de referência deve possibilitar que  repositórios de ativos de software realizem o  gerenciamento de transação, garantindo a atomicidade,  consistência, isolamento e durabilidade.
    & RF-13 
    \\ \hline

    RAS[1]
    & A arquitetura de referência de possibilitar que repositórios de  ativos de software desenvolvidos para persistir diferentes tipos  de ativos possam ser facilmente integrados.
    & RF-14
    \\ \hline

    RAS[2] 
    & A arquitetura de referência deve possibilitar que repositórios de ativos de software implementados em linguagens de  programação distintas e sob diferentes plataformas possam ser  facilmente integrados.
    & RF-15 
    \\ \hline

    RAS[3] 
    & A arquitetura de referência deve prover mecanismos para que  repositórios de ativos de software na forma de serviços possam  ser publicados e posteriormente descobertos por aplicações  cliente.
    & RN-01 
    \\ \hline
    
    RAS[4] & 
    A arquitetura de referência de prover mecanismos para que  repositórios de ativos de software orientados a serviço possam  ser compostos por processos de negócio ou utilizados por  aplicações cliente. 
    & RN-01
    
    e RN-04 
    \\ \hline

    RAS[5] & 
    A arquitetura de referência deve viabilizar o desenvolvimento  de repositórios de ativos de software que disponibilizem  informações sobre suas características e direções normativas de  uso, por meio de descrições padronizadas.
    & RF-16
    \\ \hline

    RAS[6] 
    & A arquitetura de referência deve viabilizar o desenvolvimento  de repositório de ativos de software que disponibilizem  descrições semânticas, permitindo assim sua classificação nos  repositórios de serviço.
    & RF-17
    \\ \hline

    RAS[7] 
    & A arquitetura de referência deve viabilizar o desenvolvimento  de repositório de ativos de software que tenham à disposição  informações e documentos relacionados às suas características  de qualidade. 
    & RF-18
    \\ \hline 

    RAS[8] 
    & A arquitetura de referência deve prover mecanismos para a  captura, monitoramento, registro e sinalização do não  cumprimento de requisitos de qualidade estabelecidos entre  serviços provedores e serviços clientes. 
    & RF-19
    \\ \hline 
    RAS[9] 
	& A arquitetura de referência deve viabilizar o desenvolvimento
	de repositório de ativos de software escalável, capaz de evoluir 
	de maneira incremental, por meio da composição de novas 
	funcionalidades disponíveis na forma de serviços. 
	& RN-05
	\\ \hline 

	RAS[10] 
	& A arquitetura de referência deve possibilitar que serviços de  repositório de ativos de software e composições desses  serviços sejam tratados uniformemente, ou seja, possam ser  publicados, localizados e utilizados da mesma forma. 
	& RN-01
	\\ \hline 

	RAS[11] 
	& A arquitetura de referência deve possibilitar que serviços do  repositório de ativos de software possam interagir diretamente  ou por meio do uso de barramentos de serviço. 
	& RN-01
	\\ \hline 

\end{longtable}