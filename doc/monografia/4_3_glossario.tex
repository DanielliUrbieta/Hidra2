
\glossary{name={Arquitetura de referência}, description={A estrutura e organização lógica de funcionamento do sistema computacional.}}

\glossary{name={Artefato}, description={Qualquer produto que pode ser criado, armazenado e manipulado por produtores / consumidores ou por uma ferramenta, dentro de um processo de desenvolvimento de software.}}

\glossary{name={Ativo}, description={É o elemento principal do RAS; descreve e empacota os artefatos e provê a documentação sobre como e qual problema ele resolve.}}

\glossary{name={Descritor(es)}, description={Descreve  os comportamentos e as caracteristicas do ativo.}}

\glossary{name={Repositório}, description={Representa o conceito de um armazém de ativos, que é responsável por prover um conjunto de mecanismos para a gestão e uso dos ativos}}

\glossary{name={Versionamento}, description={Criação e armazenamento de multiplas versões de um mesmo ativo.}}

\glossary{name={Web Services}, description={Recursos automatizados acessíveis pela Internet. São recursos de software ou componentes funcionais com capacidades que podem ser acessadas através de um endereço universal de Internet. Web services costumam usar XML para interagir com outros sistemas}}


\glossary{name={OMG}, description={Organização internacional que aprova padrões abertos para aplicações orientadas a objetos. Esse grupo define também a OMA (Object Management Architecture), um modelo padrão de objeto para ambientes distribuídos.}}

\glossary{name={RAS}, description={Especificação de Ativos Reusáveis}}


